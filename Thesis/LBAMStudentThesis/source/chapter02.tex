% !TeX spellcheck = en_US
\chapter{State of the art}%
\gls{am} has undergone significant advancement since its inception 25 years 
ago\cite{J.Scott.2012}. Presently, \gls{am} has achieved widespread 
utilization across industries including aerospace and dentistry. 
It demonstrates versatile capabilities for processing materials 
like metals, ceramics, polymers as well as composites\cite{Frazier.2014}.
Many researchers have classified these processing techniques into the following 
categories\cite{Kruth.1991,Hartke.2011}:

\begin{itemize}
    \item \gls{vpp}
    \item \gls{mjt}
    \item \gls{bjt}
    \item \gls{mex}
    \item \gls{pbf}
    \item \gls{shl}
    \item \gls{ded}
\end{itemize}

Each process possesses its own advantages and disadvantages, 
contingent upon factors such as the materials being processed, 
construction speed, dimensional accuracy, etc\cite{Hartke.2011}.


In this work, the primary focus is the monitoring process in \gls{pbflbm}.
Hence, in this chapter, the \gls{pbflbm} technology will be initially 
introduced, followed by an exploration of the monitoring methods employed 
in contemporary \gls{pbflbm}. Among these methods, multispectral imaging 
has been selected as the monitoring technique utilized in this work. 
Consequently, this establishes the underlying principles of the 
observation aspect within \gls{pbflbm}. Subsequently, the material's 
emissivity model will be introduced, thus leading to the current 
temperature estimation algorithms in use.
%
%
\section{Laser-based Powder Bed Fusion of Metals}


\section{Process monitoring}%

%
%
\section{Multispectral imaging}%

%
%
\section{Emissivity model}%

%
%
\section{Temperature estimation algorithm}


\section{Motivation of this thesis}