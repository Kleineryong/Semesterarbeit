% In total max. 1 Page!
\IWBstudentthesisAbstract{%
	%
	% Abstract English:
	Temperature is one of the most important parameters in \gls{pbflbm}. 
	Due to the rapid change of temperature in metallic powders upon exposure to the laser beam, 
	facilitating phase transition between liquid and solid phases. 
	The material's thermal history is relevant for the product's quality.
	So, monitoring the temperature at melt pool necessary. It is still challenging to get 
	multispectral image without interference.
	In oder to achieve this, a numerical experiment platform is formed.
	Within this virtual experimentation platform, a virtual multispectral camera has been 
	incorporated, taking into account the operational principles of actual sensors. 
	In this multispectral camera, the digital values are computed through the integration 
	of radiance intensity over wavelength. Then, hypothetical material models have been 
	developed based on various sets of raw emissivity data. Within these hypothetical material 
	models, emissivity values are configured to be wavelength and temperature-dependent, 
	also effectively simulating the phenomenon of melting observed in real materials. 
	This approach enables the emulation of emissivity characteristics akin to those found in 
	actual materials. Lastly, based on the generated experimental data, a temperature 
	estimation algorithm has been developed. This algorithm is capable of simultaneously 
	estimating the temperature and emissivity of the material using the experimental data. 
	In the conducted tests within this work, the temperature estimation algorithm with linear square 
	emissivity model has demonstrated 
	the capability to achieve temperature estimations for actual iron materials with a relative error 
	less than 3\%.
	By comparing and analyzing different temperature estimation algorithms, the importance 
	of emissivity model selection within the temperature estimation algorithm has been demonstrated. 
	Through calculations involving materials within various temperature ranges, the 
	applicability of the temperature estimation algorithm has been established.
	\newpage
}{%
	%
	% Zusammenfassung Deutsch
	Temperatur ist einer der wichtigsten Parameter im \gls{pbflbm}. Aufgrund der schnellen 
	Temperaturänderung in metallischen Pulvern bei Bestrahlung mit dem Laserstrahl kommt es zu 
	Phasenübergängen zwischen den flüssigen und festen Phasen. Die thermische Geschichte des 
	Materials ist relevant für die Qualität des Produkts.
	Daher ist die Überwachung der 
	Temperatur am Melt-pool notwendig. Es bleibt jedoch eine Herausforderung, multispektrale 
	Bilder ohne Störungen zu erhalten. Um dies zu erreichen, wurde eine numerische 
	Experimentierplattform entwickelt. Innerhalb dieser virtuellen Experimentierplattform 
	wurde eine virtuelle Multispektralkamera integriert, unter Berücksichtigung der 
	Funktionsprinzipien tatsächlicher Sensoren. In dieser Multispektralkamera werden die 
	digitalen Werte durch die Integration der Strahlungsintensität über die Wellenlänge 
	berechnet. Anschließend wurden hypothetische Materialmodelle basierend auf verschiedenen 
	Sätzen von rohen Emissionsdaten entwickelt. Innerhalb dieser hypothetischen Materialmodelle 
	werden Emissionswerte konfiguriert, die wellenlängen- und temperaturabhängig sind und 
	somit Schmelzphänomen simulieren, wie sie in realen Materialien beobachtet werden. 
	Dieser Ansatz ermöglicht die Simulation von Emissionscharakteristika, ähnlich denen in 
	tatsächlichen Materialien. Schließlich wurde basierend auf den generierten experimentellen 
	Daten ein Temperaturschätzalgorithmus entwickelt. Dieser Algorithmus ist in der Lage, 
	die Temperatur und Emissivität des Materials gleichzeitig unter Verwendung der 
	experimentellen Daten abzuschätzen. 
	In den durchgeführten Tests in dieser Arbeit hat der Temperaturabschätzalgorithmus mit 
	linearem quadratischem Emissivitätsmodell die Fähigkeit gezeigt, Temperaturabschätzungen 
	für tatsächliche Eisenmaterialien mit einem relativen Fehler von weniger als 3\% zu erzielen.
	Durch den Vergleich und die Analyse 
	verschiedener Temperaturschätzalgorithmen wurde die Bedeutung der Auswahl 
	des Emissionsmodells innerhalb des Temperaturschätzalgorithmus demonstriert. 
	Durch Berechnungen von Materialien in verschiedenen Temperaturbereichen wurde die 
	Anwendbarkeit des Temperaturschätzalgorithmus festgestellt.%
	\thispagestyle{empty}
}%
%
%