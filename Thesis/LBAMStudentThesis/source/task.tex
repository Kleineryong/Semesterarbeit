% Deckblatt
\chapter*{Scope of Work}


%\markright{Aufgabenstellung} 	% Kolumnentitel manuell auf "Aufgabenstellung"

\textbf{Title of the Semester Thesis:}\\
\Large{Investigation of multispectral imaging algorithm for temperature
determination on a numerical experiment platform in Laser-based 
Powder Bed fusion of Metals}\\
%\newline
%\normalsize{\textbf{(English Title of the Bachelor's/Master's Thesis/Semester Thesis/In\-ter\-dis\-ci\-pli\-na\-ry Project:)}}\\
%\Large{Development...}
\normalsize

\begin{tabbing}
	\hspace{7em} 		\= \hspace{13em}			\= \hspace{7em} 		\= \kill
	\textbf{Author:}  \> B.Sc. Zhaoyong Wang 	\> \textbf{Supervisor:} 	\>  M.Sc. Ruihang Dai \\
	\textbf{Issuance:} 	\> 15.03.2023 	\> \textbf{Submission:} 	\> 29.08.2023
\end{tabbing}

%\vspace{5mm}
\textbf{Setting:}\\
\gls{pbflbm} is an additive manufacturing
process which produces components with a laser by successively melting metal
powders applied in layers. This additive manufacturing process offers the advantage
of a complex lightweight design. However, complicated thermal histories
during laser-metal interaction can lead to process defects such as keyholing. Since
the process is thermally driven, acquiring the absolute temperature distribution in
the laser-material interaction zone is vital to gain a deeper understanding of the
process. However, no measurement system in \gls{pbflbm} is currently available to
determine the absolute temperature due to unknown emissivity. \gls{msi} has the 
potential to determine the absolute temperature and emissivity
simultaneously by capturing multiple wavelengths of radiance from the melt pool.


%\vspace{5mm}
\textbf{Objective:}\\
This work aims to optimize the \gls{msi} algorithm to determine the absolute temperature
accurately. Since the mapping between the digital value and radiance still
needs to be investigated, direct optimization of the \gls{msi} algorithm based on the experimental
data is challenging. Therefore, the first step is to develop a virtual experiment
to obtain "ideal" radiance determined by the given temperature distribution
and emissivity. This way, the MSI algorithm can be optimized based on the input
and calculated temperature and emissivity. The virtual experiment platform will also
allow for the investigation of various temperature distributions and emissivity models
in a time- and cost-efficient manner.%

%\vspace{5mm}
\textbf{Methodology:} \\
The content of the present thesis can be subdivided into the following tasks
\begin{itemize}
	\item Literature review regarding \gls{msi}, material emissivity model and curve fitting 
	algorithms
	\item Development of a virtual experiment platform
	\item Generate virtual experiments with various emissivity models and temperature fields
	\item Optimization of the \gls{msi} algorithm based on virtual experiments
	\item Scientific documentation of the results
\end{itemize}
% \vspace{1.0cm}

\chapter*{Declaration}
I hereby confirm that this semester thesis was written independently 
by myself without the use of any sources beyond those cited, and all
passages and ideas taken from other sources are cited accordingly.%
%
\vspace{5cm}\\
\begin{tabular}{p{0.5\linewidth}p{0.5\linewidth} }
	.....................................................		& .....................................................\\
	Location, Date  	& Signature
\end{tabular}
%
\vspace{2cm}\\
%
With the supervision of Mr. Zhaoyong Wang by Mr. Ruihang Dai intellectual property of the \gls{lbam}  flows into this work. A publication of the work or a passing on to third parties requires the permission of the head of the professorship. I agree to the archiving of the printed thesis in the  \gls{lbam} library (which is only accessible to \gls{lbam} staff) and in \gls{lbam}'s digital thesis database as a PDF document.%
%
\vfill
%
\begin{tabular}{p{0.5\linewidth}p{0.5\linewidth} }
	.....................................................		& .....................................................\\
    Location, Date  	& Signature
\end{tabular}
