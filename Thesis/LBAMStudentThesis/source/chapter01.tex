% !TeX spellcheck = en_US
\glsresetall%
\chapter{Introduction}%

\gls{am} is a different type of processing technology to conventional 
manufacturing methods. Instead of removing material from the raw part, 
additive manufactoring is usually described as "a process of joining materials
to make objects layer by layer or element by element"\cite{Frazier.2014}.
It is a useful method for reducing the the duration from product design 
to product manufacturing and imporving the material 
utilization\cite{Swift.2013b}. 


There are several \gls{am} technologies that can be sorted into different 
categories. \gls{pbflbm}, which also called \gls{sls}, is one of the powder 
based \gls{am} processes\cite{Kruth.1991}. In this process, a laser beam 
with high energy density is focused onto the surface of a metal powder bed, 
which was heated up to a temperature near the melt temperature. Then, the 
metal powder is melted and form the upper surface of the object. Last, the 
building platform will go downwards to build next layer. This process will be 
repeated until the desired object is built\cite{RevillaLeon.2019}.


Since the process is based on the phase change of materials, processing parameters 
such as laser power, layer thickness, hatch distance and scanning strategies 
are critical for producing dense materials, minimize defects, improve 
surface quality and build rate\cite{Oliveira.2020}. The common 
denominator between them is that the parameters of the process 
are varied by controlling the temperature profile of the metal powder\cite{Swift.2013b}. 
Thus, the importance of obtaining the temperature information on 
the surface layer is high. 


The solidification front of metal powders moves with a high velocity (generally 
$0.01 \sim 10 \, \text{m/s}$)\cite{DebRoy.2018}, the cooling rate of material 
is normally at range of $10^5 \sim 10^6 \, \text{K/m}$\cite{Oliveira.2020}. 
Thus, a contactless temperature measurement is required to obtain the temperature 
of the surface area. 


Two different methods is used for monitoring the surface temperature 
in recent studies, namely conventional infrared irradiation 
temperature measurement methods and multi-wavelength techniques\cite{Li.2019}.
Since 

\section{Multispectral imaging}

\section{Curve fit algorithms}
