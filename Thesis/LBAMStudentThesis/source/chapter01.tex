% !TeX spellcheck = en_US
\glsresetall%
\chapter{Introduction}%

\gls{am} is a different type of processing technology to conventional 
manufacturing methods. Instead of removing material from the raw part, 
additive manufactoring is usually described as "a process of joining materials
to make objects layer by layer or element by element"\cite{Frazier.2014}.
It is a useful method for reducing the the duration from product design 
to product manufacturing and imporving the material 
utilization\cite{Swift.2013b}. 


There are several \gls{am} technologies that can be sorted into different 
categories. \gls{pbflbm}, which also called \gls{sls}, is one of the powder 
based \gls{am} processes\cite{Kruth.1991}. In this process, a laser beam 
with high energy density is focused onto the surface of a metal powder bed, 
which was heated up to a temperature near the melt temperature. Then, the 
metal powder is melted and form the upper surface of the object. Last, the 
building platform will go downwards to build next layer. This process will be 
repeated until the desired object is built\cite{RevillaLeon.2019}.


Since the 
%
\section{Processing parameters}

\section{Multispectral imaging}

\section{Curve fit algorithms}

\section{Motivation of this thesis}