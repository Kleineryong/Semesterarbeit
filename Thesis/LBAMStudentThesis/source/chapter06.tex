\chapter{Conclusion and prospects}
\section*{Conslusion}
To facilitate temperature monitoring during the processing of \gls{pbflbm}, multispectral cameras are 
employed to capture the thermal history of the melt pool. In this study, a virtual experiment 
platform was developed to obtain undisturbed experimental data with a low cost. Within this virtual experimental 
platform, a virtual multispectral camera were configured to obtain the experiment data from hypothetical materials.

When designing the virtual experimental platform, several considerations were taken into account:

\begin{enumerate}
    \item Imperfections of camera sensor and lens system: real multispectral 
    cameras use pixel-level filters, whose properties can be thought of as intrinsic to the 
    sensor. These filters are not ideal and allow radiation from wavelengths beyond those 
    intended to pass through. This characteristic is characterized by the filter's amplitude-frequency 
    response curve, referred to as the quantum efficiency. In this study, each channel of the 
    virtual multispectral camera has one wavelength-related quantum efficiency.
    This feature is employed to emulate the performance of a real multispectral camera.
    
    %\item Imperfections of lens system: the imperfections of the camera's lens system also
    %play an important role in the overall performance. As radiation needs to pass through the lens 
    %system before reaching the sensor surface, it's important to account for these effects. 
    %In this study, the wavelength-related transparency of the lens system has been incorporated 
    %into the virtual experimental platform. This inclusion aims to simulate the non-ideal optical 
    %conditions that exist in real experiments.

    \item Principle of data acquisition: the digital value obtained for each pixel on the 
    camera's sensor is directly related to the quantity of protons that land on it during 
    the exposure time\cite{JamesC.Mullikin.1994}. Consequently, the data obtained from the 
    camera's sensor represents an integration of the received radiation intensity 
    within a certain wavelength range. Therefore, in this study, the virtual 
    multispectral camera in the virtual experimental platform employs an integration 
    method to calculate the digital value of radiation. This approach ensures that the 
    simulated data capture process mirrors the physical behavior of an actual multispectral 
    camera sensor, providing a realistic basis for temperature monitoring study during 
    the \gls{pbflbm} process.

    \item The emissivity model of the hypothetical material: In real conditions, a material's 
    emissivity is influenced by various factors such as wavelength and temperature. To simulate the 
    emissivity characteristics of real materials, material models based on different raw
    wavelength-realted emissivity data were developed. Within these material models, a temperature factor was 
    employed to describe how emissivity changes with temperature. Simultaneously, these 
    models also account for the emissivity characteristics of materials during phase transitions.
    % By incorporating temperature and wavelength-dependent emissivity behavior and accounting 
    % for phase transitions, the virtual experimental platform is now equipped with 
    % hypothetical materials that exhibit comparable emissivity properties to real materials. 

\end{enumerate}


After obtaining experimental data within the virtual experiment platform, a temperature estimation 
algorithm is applied to estimate the temperature and emissivity of the materials based on the 
acquired data.


The temperature estimation algorithm is based on a curve fitting approach. It reconstructs 
the blackbody radiation and emissivity model to fit the experiment data using a curve fitting 
algorithm. As a result, five different wavelength-related emissivity models are introduced for temperature 
estimation: linear model, linear square model, quadratic model, exponential model and mixed model. 


After applying these models, the algorithm analyzes the results obtained from each of them. 
This analysis involves comparing the estimated temperature values to known reference values, 
assessing the fit quality and consistency of the reconstructed curves, and evaluating the overall performance 
of each emissivity model.

\begin{enumerate}

\item The emissivity model plays a crucial role in the performance of the temperature estimation algorithm. It impacts the accuracy and consistency of the program's execution. Consequently, when performing temperature estimation for materials, it is imperative to select an appropriate emissivity model.


\item The utilization of a mixed model holds the potential to facilitate fitting more intricate radiation behaviors. When the emissivity model is appropriately chosen, it can yield optimal computational performance, without compromising the consistency of computational operations for programmers,

\item In principle, the temperature estimation algorithm is feasible. However, it requires a specialized emissivity model. Due to the variations in spectral radiation profiles at elevated temperatures, the conventional emissivity model introduces additional perturbations. Hence, an additional emissivity model is necessary to achieve accurate temperature estimation for high-temperature materials.
\end{enumerate}

So, the influence of the emissivity model on the performance of the temperature 
estimation algorithm is demonstrated. As mentioned earlier, the choice of an appropriate 
emissivity model plays a crucial role in obtaining accurate temperature estimations.

\newpage
\section*{Prospect}
The primary objective of establishing the virtual experimental platform in this study is to 
simulate the \gls{pbflbm} process, enabling the acquisition of noise-free multispectral 
images at a relatively low cost. This platform serves as a valuable tool for refining 
the temperature estimation algorithm. However, there are several areas within the virtual 
experimental platform that could benefit from improvement.


Firstly, an enhanced material model should be introduced. Due to the limited availability 
of accurate emissivity data for materials, the hypothetical material utilized in the virtual 
experiment platform cannot perfectly replicate the experimental conditions of real materials. 
Hence, the incorporation of a more realistic material model would significantly contribute to 
enhancing the authenticity of the virtual experimentation platform.


Secondly, the emissivity model used in temperature estimation algorithm can be further 
improved. As mentioned earlier, the inclusion of additional parameter in the linear square 
model significantly enhanced the performance of the temperature estimation algorithm. 
Consequently, for more complex real materials, it is reasonable to expect the development of 
more suitable emissivity models. This approach holds the potential to not only reduce 
computational time but also enhance the accuracy and consistency of the calculations.


Subsequently, the temperature estimation algorithm itself is considered. The temperature 
estimation algorithm employed in this study is designed for single-point temperature estimation, 
lacking the ability to capture temperature information between adjacent points. 
However, in real systems, due to heat and mass transfer phenomenon, temperatures between 
neighboring points are not independent of each other. Therefore, this information can also be 
leveraged to enhance the precision of the temperature estimation algorithm.




